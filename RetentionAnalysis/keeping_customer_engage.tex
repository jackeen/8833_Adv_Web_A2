\subsection{Keeping Customers Engaged}
In tourism industry, customer interaction is challenged by low purchase frequency—months or even years may pass between trips. This “quiet period” is when customer relationships are most vulnerable, and brands risk being forgotten or replaced \cite{huang2022enhancing}. Maintaining meaningful engagement during these gaps is crucial for retention.

\subsubsection{Systematic Post-Trip Follow-Up and Feedback Loops}
Description: CRM systems should automatically initiate a structured communication sequence post-trip, typically including: (1) “Welcome Home” email shortly after return, (2) Thank-you message with a feedback survey. (3) User-generated content (UGC) invitation to share photos or stories on social media using a branded hashtag.

Importance: These simple actions convey a powerful message: the agency cares about the entire journey, not just the sale. This builds trust \cite{dellacorte2015customer}. Feedback also provides invaluable insights for improving service quality \cite{kim2022impact}. Moreover, UGC serves as compelling social proof. \cite{xu2021understanding} found that customers trust UGC far more than brand-produced content.

\subsubsection{Fostering Online Brand Communities}
Description: Create a dedicated online community—on the agency's website or social platforms—where past and future travelers can interact, share tips, post reviews, ask questions, find travel buddies, and communicate directly with staff.
Importance: Online communities are powerful incubators for deep brand loyalty. They foster a sense of belonging and shared emotion that transcends commercial ties. Community interaction strengthens trust and commitment \cite{guan2022net}. A well-run, high-quality community makes users loyal to the platform, which eventually translates into brand loyalty. In this way, a fragmented customer base can become a cohesive, active, and deeply loyal brand tribe \cite{bui2014importance}. 
