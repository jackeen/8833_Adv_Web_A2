\section{Customer Retention in Tourism Industry}
Customer retention is the strategic process to maintain the long-term relationships between customers and business for customers' loyalty \cite{Sharma2023CustomerRelationship}. It enhances profitability by increasing purchases and reducing customer churn, which is critical in the competitive tourism sector. It enables customer relationship management (CRM) data to strengthen customer trust and emotional connection to the brand \cite{So2023ServiceQuality}.

% importance
Customer retention is critical in the tourism industry due to its competitive bringing. High retention rate reduces customers' costs and businesses' profitability. CRM systems play a key role by enabling personalized interactions and services, thereby, it strengthen long-term relationships. \cite{Sharma2023CustomerRelationship} emphasize that the effective CRM strategies enhance retention by leveraging data.

% Strategy Challenges
The strategies in tourism include personalized communication and proactive service recovery. Proactive activities through social media campaigns and travel inspiration content ensures customers' connecting to the brand. \cite{Kim2023LoyaltyPrograms} highlight that loyalty programs, supported by CRM analytics, significantly enhance retention by emotional attachment. \cite{Shafiee2023ServiceRecovery} note that over-reliance on automated CRM causes risks lacking empathy and reducing retention effectiveness. 

% next parts
% Building a relationship with customers
% Offering incentives
% Keeping customers engaged
% Addressing customer concerns
% Providing quality products or services

% sub modules for each part
% definition
% importance in Tourism
% strategy
% Challenges
% Practical Implications
% Recommendations

% possible lack of functionalities in current CRM system
% 1 training system for staff
% 1 instant communication and variety supporting
% 0 sentiment analysis

