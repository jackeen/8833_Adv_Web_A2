\subsection{Offering Incentives}
If building relationships is the foundation of customer retention, then incentives are the catalyst that drives repeat visits and deepens loyalty. Well-designed incentive programs effectively shape customer behavior by increasing their “switching costs,” making them more likely to stay with a brand rather than switch to competitors \cite{lian2021effects}. In a competitive market like tourism, they are often seen as the "price of entry." According to \cite{expedia2017new}, 58\% of travelers consider loyalty programs when making travel decisions.

\subsubsection{Implementing Tiered Loyalty Programs}
Description: This is a structured membership system where the CRM automatically assigns customers to different tiers (e.g., Bronze, Silver, Gold, Platinum) based on their spending, purchase frequency, or level of engagement. Higher tiers offer more exclusive benefits such as better discounts, priority booking, free upgrades, and dedicated customer service.

Importance: These programs tap into customers' desire for status, achievement, and exclusivity. The appeal of better perks encourages customers to concentrate their travel spending with a single agency to attain higher tiers \cite{rahman2022effect}. In a study on hotel loyalty programs, \cite{tanford2013impact} found that behavioral loyalty—measured by the percentage of nights booked with a preferred brand—increased significantly with tier level: 53\% for basic members, 66\% for mid-tier, and 78\% for elite members.

\subsubsection{Provide Personalized and Flexible Rewards}
Description: Travel agencies should leverage CRM data to offer rewards that are highly relevant to individual customers. This approach aligns with the concept of “fourth-level loyalty programs,” where incentives are customized based on a customer's purchase history \cite{chetty2020loyalty}. For example, if the CRM identifies a customer as a family traveler, the agency could offer a complimentary upgrade to a family suite on their next booking. Meanwhile, for a solo traveler with a passion for adventure, a free guided hiking experience could be an ideal reward.

Importance: Personalized rewards are far more impactful than standard discounts. They show that the agency truly understands each customer's lifestyle and preferences, reinforcing the emotional connection between the customer and the brand. This signals that the agency is adapting to the customer—rather than forcing the customer to adapt to a rigid system—which in itself is a powerful tool for relationship maintenance \cite{othman2025}.

\subsubsection{Experiential Rewards and Incentive Travel}
Description: For top-value clients, offering non-material, unique “money-can't-buy” experiences is the ultimate loyalty strategy. The most exclusive form is incentive travel—an all-expenses-paid, carefully curated travel experience \cite{whalen2017engaging}.

Importance: Compared to cash, travel rewards deliver much stronger emotional impact. While cash is quickly spent and forgotten, a memorable travel experience creates lasting, emotionally-charged brand associations. Rose (2025) noted that 100\% of “best-in-class” companies (with the highest retention and growth rates) use group travel as an incentive. The goal is to make top clients feel deeply valued and emotionally connected to the brand, beyond mere transactions \cite{huang2022enhancing}.
